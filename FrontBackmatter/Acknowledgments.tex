%*******************************************************
% Acknowledgments
%*******************************************************
\pdfbookmark[1]{Acknowledgments}{acknowledgments}

% \begin{flushright}{\slshape    
%     ``Marry, and you will regret it; don’t marry, you will also regret it; marry or don’t marry, you will regret it either way.
%     Laugh at the world’s foolishness, you will regret it; weep over it, you will regret that too; laugh at the world’s foolishness or weep over it, you will regret both.
%     Believe a woman, you will regret it; believe her not, you will also regret it...
%     Hang yourself, you will regret it; do not hang yourself, and you will regret that too; hang yourself or don’t hang yourself, you’ll regret it either way; whether you hang yourself or do not hang yourself, you will regret both.
%     This, gentlemen, is the essence of all philosophy.''} \\ \medskip
%     --- \defcitealias{Kierkegaard:1843}{Sören Kierkegaard}\citetalias{Kierkegaard:1843} \citep{Kierkegaard:1843}
% \end{flushright}

\bigskip

\begingroup
\let\clearpage\relax
\let\cleardoublepage\relax
\let\cleardoublepage\relax

\chapter*{Acknowledgments}
\label{ch:acknowledgments}

This thesis would not have been possible without the invaluable guidance of Professor Frédéric Blanqui,
who introduced me to Lambdapi and implemented several modifications to the tool that were essential for the reconstruction of SMT proofs.
Working with him on Lambdapi has also instilled in me a strong desire to contribute to its long-term development.
I am also profoundly grateful to him for welcoming me into the EuroProofNet research network, which offered me many opportunities for collaboration and for presenting my work.

\bigskip

I would like to express my deepest appreciation to Professor Haniel Barbosa for sharing his deep expertise in SMT solvers and SMT proof formats
, and for the strong interest he has consistently shown in my work, which I greatly appreciated.
I am equally indebted to his student, Bruno Andreotti, the main developer of Carcara, for his constant availability, generosity in our exchanges, and for his unfailing reactivity to my many requests.

\bigskip

I wish to extend my sincere thanks to Professor Jean-Paul Bodeveix, who has supported and followed my academic path since my Master's studies.
It has been a pleasure and an honour to collaborate with him in the context of the ANR ICSPA project.

\bigskip

I would also like to thank the members of the ICSPA project for their support and for the many fruitful discussions we shared.
I am also particularly grateful to Mathias Fleury, Hanna Lachnitt, Chantal Keller, and Hans-Jörg Schurr, with whom I had the pleasure of working and exchanging ideas.
I would also like to thank the members of the VerDis team, especially my four ``thesis siblings'': Thomas Bagrel, Ghilain Bergeron, Sarah Depernet, and Vincent Trélat, for their support, discussions, and companionship.
I am especially grateful to Sophie Tourret for including me in the Carma project, which greatly facilitated the development of this work in collaboration with Professor Haniel Barbosa and Bruno Andreotti.
I also wish to thank the members of the Deducteam, in particular Theo Winterhalter, Ciaran Dune, Melanie Tapproge, Thomas Traversié, Rishikesh Vaishnav, Nicolas Margulies and Bruno Barras, for the stimulating scientific environment they helped create.

\smallskip

I am deeply indebted to my family, particularly to my parents, for their unwavering support throughout these years. I would also like to warmly thank my former colleague Didier Plaindoux for his encouragement over the course of my studies and research.
Finally, I would like to thank Xavier Van de Woestyne for creating the image used on the cover of this thesis, which gives a distinctive visual identity to this work.

\bigskip

I would also like to express my sincere gratitude to the ANR ICSPA project directed by Professor Catherine Dubois for funding my PhD.
Its financial support made this research possible and provided the conditions necessary for me to carry out this work in a stimulating and collaborative scientific environment.


\endgroup



