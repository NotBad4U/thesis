%*****************************************
\chapter{Reconstructing SMT Proofs}\label{ch:reconstruction}
%*****************************************

\section{Translating the SMT input problem}

We now describe how we encode input problems expressed in a given
SMT-LIB signature \cite[\S 5.2.1]{smtlib}. In order to avoid a notational clash with the Lambdapi signature $\Sigma$, we denote the set of SMT-LIB sorts as $\Theta^\mathcal{S}$, the set of function symbols $\Theta^\cal{F}$, the set of natural numbers $\Theta^\cal{N}$, and the set of variables $\Theta^\cal{X}$.
Alethe does not support the sorts \texttt{Array} and \texttt{String}. Our translation is based on the following functions:
\begin{itemize}
\item $\cal{D}$ translates declarations of sorts and functions in $\Theta^\cal{S}$ and $\Theta^\mathcal{F}$ into constants,
\item $\cal{S}$ maps sorts to $\Sigma$ types,
\item $\cal{E}$ translates SMT expression to $\lpm$ terms,
\item $\cal{C}$ translates a list of commands  $c_1 \dots c_n$ of the form\\
  $i.~\Gamma \triangleright~\varphi~(\mathcal{R}~P)[A]$ to typing judgments $\Gamma \vdash_\Sigma i : N$ with $N$ the corresponding type of the clause $\varphi$.
\end{itemize}

\smallskip

\begin{definition}[Function $\mathcal{D}$ translating SMT sort and function symbol declarations]
For each sort symbol $s$ with arity $n$ in $\Theta^\cal{S}$ we create a constant $s: \set \ra \dots \ra \set$.
For each function symbol $f~\sigma^+$ in $\Theta^\cal{F}$ we create a constant $f: \cal{S}(\sigma^+)$.
\label{def:function-d}
\end{definition}

\smallskip

In other words, all SMT sorts used in the Alethe proof trace will be defined as constants that inhabit the type \set{} in the signature context $\Sigma$.
For every function declared in the SMT prelude, we define a constant whose arity follows the sort declared in the SMT prelude. The translation of sorts is formally defined as follows.

\smallskip

\begin{definition}[Function $\mathcal{S}$ translating sorts of expression] 
  The definition of $\mathcal{S}$(s) is as follows.
  \begin{itemize}
    \item Case $s =$ \smtinline{Bool}, then $\Sort{s} = \el\,o$,
    \item Case $s =$ \smtinline{Int}, then $\Sort{s} = \el~\tt{int}$,
    \item Case $s =$ \smtinline{Real}, then $\Sort{s} = \el\,\tt{rational}$,
    \item Case $s =$ \smtinline{BitVec} $n$, then $\Sort{s} = \el\,(\tt{bitvec}~\cal{E}(n))$,
    \item Case $s = \sigma_1\,\sigma_2 \dots \sigma_n$ then $\Sort{s} = \el{} (\mathcal{S}(\sigma_1) \leadsto \dots \leadsto \mathcal{S}(\sigma_n))$,
    \item otherwise $\Sort{s} = \el\, \mathcal{D}(s)$.
    % where the symbol $s$ on the right-hand side denotes the Lambdapi sort introduced for the SMT sort $s$.
  \end{itemize}
  \label{def:function-s}
\end{definition}

\smallskip

\begin{example}{Translation of the prelude in \cref{lst:smtexampleinput}.}
\begin{lstlisting}[language=Lambdapi]
symbol U : Set;
symbol a : El U;
symbol b : El U;
symbol p : El (U ⤳ o);
\end{lstlisting}
\end{example}

\smallskip

\begin{definition}[Function $\cal{E}$ translating SMT expressions]
The definition of $\E{e}$ is as follows.
\begin{itemize}
\setlength{\parskip}{0pt}
\item Case $e = c$, with $c \in \cal{N}$ then $\E{c} = c$
\item Case $e = (u \slash d)$, with $u,d \in \Theta^\cal{N}$ then $\E{c} = \E{u} \,\#\, \E{d}$
\item Case $e =$ \smtinline{true}, then $\E{e} = \top$
\item Case $e =$ \smtinline{false}, then $\E{e} = \bot$
\item Case $e = (p~t_1~t_2\dots~t_n)$ and $p$ an logic operator, then $\E{e} = \E{t_1}~p~\dots~p~\E{t_n}$.
\item Case $e = (p~t_1~t_2\dots~t_n)$ and $p$ an operator, then $\E{e} = \E{t_1}~p~\dots~p~\E{t_n}$.
\item Case $e = (f~t_1\dots~t_n)$ with $f \in \Theta^\cal{F}$,\\
  then $\E{e} = (\mathcal{D}(f)~\E{t_1}~\dots~\E{t_n})$.
\item Case $e = (\approx~t_1~t_2)$ then $\E{e} = (\E{t_1} = \E{t_2})$.
\item Case $e = (Q~x_1  \dots x_n ~t)$ and $Q\in \{\kw{forall}, \kw{exists}\}$\\
  then $\E{e} = Q x_1, \dots, Q x_n, \E{t}$. 
\item Case $e = (\kw{choice}~x : \sigma ~t)$ then $\E{e} = \epsilon~x: \cal{S}(\sigma),\, \E{t}$.
\item Case $e = (x: \sigma )$ with $x \in \Theta^\mathcal{X}$ a sorted variable, then $\E{e} = x: \cal{S}(\sigma)$.
\item Case $e = (\kw{ite}~c~t~e)$, then $\E{e} = \kw{ite}~\E{c}~\E{t}~\E{e}$.
\item Case $e = (\kw{xor}~a~b)$, then $\E{e} = \kw{xor}~\E{a}~\E{b}$.
\item Case $e = (\kw{distinct}~t_1 \dots t_n)$,\\
  then $\E{e} = \kw{distinct}~(\E{t_1} \colon\colon_\bb{V} \dots\, \colon\colon_\bb{V} ~\E{b} \colon\colon_\bb{V} \Box_\bb{V})$.
\item Case $e = (\kw{bbt}~x_1 \dots x_n)$, then $\E{e} = \mathop{bbt}~p~(\E{x_1} \mathbin{\colon\colon} \dots \mathbin{\colon\colon} \E{x_n} \mathbin{\colon\colon} \square)$
with $p$ a proof that \((\E{x_1} \mathbin{\colon\colon} \dots \mathbin{\colon\colon} \E{x_n} \mathbin{\colon\colon} \square)\) is well-formed.
\end{itemize}
\label{def:function-e}
\end{definition}

We define the function $\cal{C}$ in \cref{def:function-c} that translates an Alethe step of the form $i.~\Gamma~\triangleright~l_1 \dots l_n\,(R\,P)[A]$ into a constant $i: \pid (\E{l_1}\cons\dots\cons\E{l_n}\cons \nil) \coloneq M$ where $M$ is a proof term of appropriate type.
We recall that an Alethe proof trace is a list $c_1 \dots c_m$ of steps, where every step may only refer to previous steps by index in its justification. Therefore, we can translate each step independently.
The function $\cal{C}$ is defined by cases on the rule $R$. In the case of an \texttt{assume} command, we do not provide a proof term $M$ so $i$ is considered as an axiom. Because a step can only refer to previous steps, each step $i$ can be translated separately. Its proof term may refer to constants representing steps that precede $i$ in the trace and that must therefore have been introduced when Lambdapi checks the definition of constant~$i$.

\begin{definition}[Function $\cal{C}$ translating step]
Given a step 
\[
  i. \, \Gamma \, \triangleright\, l_1 \dots l_n\,(R\,P)[A]
\]
the function $\cal{C}(i)$ produce a constant
\[
  \cal{C}(i) = i: \pid (\E{l_1}\cons\dots\cons\E{l_n}\cons \nil) \coloneq M
\]
\label{def:function-c}
\end{definition}

In the following sections, we present how different categories of rules are translated by $\cal{C}$. We will sometimes omit writing $\pid$ and $\prf$ in order to simplify the notation.