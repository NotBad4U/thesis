%********************************************************************
% Appendix
%*******************************************************
% If problems with the headers: get headings in appendix etc. right
%\markboth{\spacedlowsmallcaps{Appendix}}{\spacedlowsmallcaps{Appendix}}
\chapter{Appendix}

\section{Confluence of the rewriting rules of integers and positive binary number}
\label{app:confluence-int-pos}

The rules presented below represent the relations $\ra_\bb{Z}$ and $\ra_\bb{P}$ encoded in the TRS\footnote{\url{http://www.lri.fr/~marche/tpdb/format.html}} format accepted by the \cite{CSI} tool.
These rules can be used to rerun the tool in order to verify the confluence property.


\begin{lstlisting}[language=trs, caption=Rewriting rule of $\bb{Z}$ and $\bb{P}$ in the TRS format]
(VAR
  a: Z
  b: Z
  x : P
  q : P
  y : P
)
(RULES
  ~(Z0) -> Z0
  ~(Zpos(p)) -> Zneg(p)
  ~(Zneg(p)) -> Zpos(p)
  ~(~(a)) -> a
  ~(add(a,b)) -> add(~(a), ~(b))

  double(Z0) -> Z0
  double(Zpos(p)) -> Zpos(O(p))
  double(Zneg(p)) -> Zneg(O(p))
  
  succ_double(Z0) -> Zpos(H)
  succ_double(Zpos(p)) -> Zpos(I(p))
  succ_double(Zneg(p)) -> Zneg(pos_pred_double(p))
  
  pred_double(Z0) -> Zneg(H)
  pred_double(Zpos(p)) -> Zpos(pos_pred_double(p))
  pred_double(Zneg(p)) -> Zneg(I(p))

  sub(I(p), I(q)) -> double(sub(p, q))
  sub(I(p), O(q)) -> succ_double(sub(p, q))
  sub(I(p), H) -> Zpos(O(p))
  sub(O(p), I(q)) -> pred_double(sub(p, q))
  sub(O(p), O(q)) -> double(sub(p, q))
  sub(O(p), H) -> Zpos(pos_pred_double(p))
  sub(H, I(q)) -> Zneg(O(q))
  sub(H, O(q)) -> Zneg(pos_pred_double(q))
  sub(H, H) -> Z0

  +(Z0,a) -> a
  +(a,Z0) -> a
  +(Zpos(x), Zpos(y)) -> Zpos(add(x, y))
  +(Zpos(x), Zneg(y)) -> sub(x, y)
  +(Zneg(x), Zpos(y)) -> sub(y, x)
  +(Zneg(x), Zneg(y)) -> Zneg(add(x, y))
  
  mult(Z0, a) -> Z0
  mult(a, Z0) -> Z0
  mult(Zpos(x), Zpos(y)) -> Zpos(mul(x, y))
  mult(Zpos(x), Zneg(y)) -> Zneg(mul(x, y))
  mult(Zneg(x), Zpos(y)) -> Zneg(mul(x, y))
  mult(Zneg(x), Zneg(y)) -> Zpos(mul(x, y))


  succ(I(x)) -> O(succ(x))
  succ(O(x)) -> I(x)
  succ(H) -> O(H)
  add(I(x), I(q)) -> O(addcarry(x, q))
  add(I(x), O(q)) -> I(add(x, q))
  add(O(x), I(q)) -> I(add(x, q))
  add(O(x), O(q)) -> O(add(x, q))
  add(x, H) -> succ(x)
  add(H, y) -> succ(y)

  addcarry(I(x), I(q)) -> I(addcarry(x, q))
  addcarry(I(x), O(q)) -> O(addcarry(x, q))
  addcarry(O(x), I(q)) -> O(addcarry(x, q))
  addcarry(O(x), O(q)) -> I(add(x, q))
  addcarry(x, H) -> add(x, O(H))
  addcarry(H, y) -> add(O(H), y)
  
  pos_pred_double(I(x)) -> I(O(x))
  pos_pred_double(O(x)) -> I(pos_pred_double(x))
  pos_pred_double(H) -> H
  
  mul(I(x), y) -> add(x, O(mul(x,y)))
  mul(O(x), y) -> O(mul(x, y))
  mul(H, y) -> y
)
\end{lstlisting}

\section{Arithmetic definitions}\label{app:arith-def}
\label{app:arith-defs}

\begin{definition}
{\footnotesize
% \centering
% --- Integer Addition ---
\begin{align*}
&+: \bb{Z} \ra \bb{Z} \ra \bb{Z} \\
& \ZO + y \re y \\
& x + \ZO \re x \\
& (\tt{Zpos x}) + (\tt{Zpos y}) \re (\ZPos~(\tt{add}~x~y))  \\
& (\tt{Zpos x}) + (\tt{Zneg y}) \re (\tt{sub}~x~y)  \\
& (\tt{Zneg x}) + (\tt{Zpos y}) \re (\tt{sub}~y~x)  \\
& (\tt{Zneg x}) + (\tt{Zneg y}) \re \ZNeg~(\tt{add}~x~y)
\end{align*}
\noindent

\begin{align*}
& \kw{succ}: \bb{P} \ra \bb{P} \\
& \kw{succ}~(I~x) \re O (\kw{succ}~x) \\
& \kw{succ}~(O~x) \re I (\kw{succ}~x) \\
& \kw{succ}~(H) \re O H
\end{align*}

% --- Positive Binary Arithmetic ---
\[
\begin{array}[t]{ll}
\begin{aligned}
&\tt{add} : \bb{P} \ra \bb{P} \ra \bb{P} \\
& \tt{add}~(\tt{I}~x)~(\tt{I}~q) \re \tt{O}~(\tt{addc}~x~q) \\
& \tt{add}~(\tt{I}~x)~(\tt{O}~q) \re \tt{I}~(\tt{add}~x~q) \\
& \tt{add}~(\tt{O}~x)~(\tt{I}~q) \re \tt{I}~(\tt{add}~x~q) \\
& \tt{add}~(\tt{O}~x)~(\tt{O}~q) \re \tt{O}~(\tt{add}~x~q) \\
& \tt{add}~x~\tt{H} \re \tt{succ}~x \\
& \tt{add}~\tt{H}~y \re \tt{succ}~y \\
\end{aligned}
&
\begin{aligned}
&\tt{mul} : \bb{P} \ra \bb{P} \ra \bb{P} \\
& \tt{mul}~\tt{H}~y \re y \\
& \tt{mul}~y~\tt{H} \re y \\
& \tt{mul}~(\tt{O}~x)~y \re \tt{O}~(\tt{mul}~x~y) \\
& \tt{mul}~(\tt{I}~x)~y \re \\
& \quad \tt{add}~y~(\tt{O}~(\tt{mul}~x~y)) \\
\end{aligned}
\end{array}
\]
\noindent

\begin{align*}
&\tt{addc} : \bb{P} \ra \bb{P} \ra \bb{P} \\
& \tt{addc}~(\tt{I}~x)~(\tt{I}~q) \re \tt{I}~(\tt{addc}~x~q) \\
& \tt{addc}~(\tt{I}~x)~(\tt{O}~q) \re \tt{O}~(\tt{addc}~x~q) \\
& \tt{addc}~(\tt{O}~x)~(\tt{I}~q) \re \tt{O}~(\tt{addc}~x~q) \\
& \tt{addc}~(\tt{O}~x)~(\tt{O}~q) \re \tt{I}~(\tt{add}~x~q) \\
& \tt{addc}~x~\tt{H} \re \tt{add}~x~(\tt{O}~\tt{H}) \\
& \tt{addc}~\tt{H}~y \re \tt{add}~(\tt{O}~\tt{H})~y \\
\end{align*}
\noindent

% --- Multiplication ---
\begin{align*}
&\tt{*} : \bb{Z} \ra \bb{Z} \ra \bb{Z} \\
& \ZO *~\_ \re \ZO \\
& \_ *~\ZO \re \ZO \\
& \ZPos x *~\ZPos y \re \ZPos (\tt{mul}~x~y) \\
& \ZPos x *~\ZNeg    y \re \ZNeg    (\tt{mul}~x~y) \\
& \ZNeg    x *~\ZPos y \re \ZNeg    (\tt{mul}~x~y) \\
& \ZNeg    x *~\ZNeg    y \re \ZPos (\tt{mul}~x~y) \\
\end{align*}
}%
\end{definition}

\begin{definition}[Complete definition of \tt{sub} function]
\begin{align*}
& \kw{sub} : \bb{P} \ra \bb{P} \ra \bb{Z} \\
& \kw{sub}~(I~p)~(I~q) \re \kw{double}~(\kw{sub}~p~q) \\
& \kw{sub}~(I~p)~(O~q) \re \kw{succ\_double}~(\kw{sub}~p~q) \\
& \kw{sub}~(I~p)~H \re \kw{Zpos}~(O~p) \\
& \kw{sub}~(O~p)~(I~q) \re \kw{pred\_double}~(\kw{sub}~p~q) \\
& \kw{sub}~(O~p)~(O~q) \re \kw{double}~(\kw{sub}~p~q) \\
& \kw{sub}~(O~p)~H \re \kw{Zpos}~(\kw{pos\_pred\_double}~p) \\
& \kw{sub}~H~(I~q) \re \kw{Zneg}~(O~q) \\
& \kw{sub}~H~(O~q) \re \kw{Zneg}~(\kw{pos\_pred\_double}~q) \\
& \kw{sub}~H~H \re \kw{Z0} \\[1em]
%
& \kw{pos\_pred\_double} : \bb{P} \ra \bb{P} \\
& \kw{pos\_pred\_double}~(I~x) \re I~(O~x) \\
& \kw{pos\_pred\_double}~(O~x) \re I~(\kw{pos\_pred\_double}~x) \\
& \kw{pos\_pred\_double}~H \re H \\[1em]
%
& \kw{double} : \bb{Z} \ra \bb{Z} \\
& \kw{double}~Z0 \re Z0 \\
& \kw{double}~(\kw{Zpos}~p) \re \kw{Zpos}~(O~p) \\
& \kw{double}~(\kw{Zneg}~p) \re \kw{Zneg}~(O~p) \\[1em]
%
& \kw{succ\_double} : \bb{Z} \ra \bb{Z} \\
& \kw{succ\_double}~Z0 \re \kw{Zpos}~H \\
& \kw{succ\_double}~(\kw{Zpos}~p) \re \kw{Zpos}~(I~p) \\
& \kw{succ\_double}~(\kw{Zneg}~p) \re \kw{Zneg}~(\kw{pos\_pred\_double}~p) \\[1em]
%
& \kw{pred\_double} : \bb{Z} \ra \bb{Z} \\
& \kw{pred\_double}~Z0 \re \kw{Zneg}~H \\
& \kw{pred\_double}~(\kw{Zpos}~p) \re \kw{Zpos}~(\kw{pos\_pred\_double}~p) \\
& \kw{pred\_double}~(\kw{Zneg}~p) \re \kw{Zneg}~(I~p)
\end{align*}
\end{definition}

\section{Supported Alethe rules}
\label{app:alethe-rules-supported}

% \begin{figure}[H]
% \centering
% \begin{tabular}{|cccc|}
% \hline
% \kw{assume} & \kw{subproof} & \kw{resolution} & \kw{th\_resolution}  \\
% \kw{true} & \kw{false} & \kw{not\_not} & \kw{forall\_inst}   \\
% \kw{refl} & \kw{eq\_reflexive} & \kw{eq\_transitive} & \kw{eq\_congruent} \\
% \kw{eq\_congruent\_pred} & \kw{and\_pos} & \kw{and\_neg} & \kw{or\_pos} \\
% \kw{or\_neg} & \kw{xor\_pos1} & \kw{xor\_pos2} & \kw{xor\_neg1} \\
% \kw{xor\_neg2}  & \kw{xor\_pos1} & \kw{contraction} & \kw{bind} \\
% \kw{implies\_pos} & \kw{implies\_neg1}  & \kw{implies\_neg2} & \kw{equiv\_pos1} \\
% \kw{equiv\_pos2} & \kw{equiv\_neg1}  & \kw{equiv\_neg2} & \kw{ite\_pos1} \\
% \kw{ite\_pos1} & \kw{ite\_pos2} & \kw{ite\_neg1} & \kw{ite\_neg2}  \\
% \kw{distinct\_elim} & \kw{eq\_symmetric} & \kw{sko\_exist} & \kw{sko\_forall} \\
% \kw{and} & \kw{not\_or} & \kw{not\_and} & \kw{reordering} \\
% \kw{symm} & \kw{not\_symm} & \kw{eq\_symmetric} & \kw{equiv\_simplify} \\
% \kw{not\_simplify} & \kw{ite\_simplify} & \kw{implies\_simplify} & \kw{ac\_simp} \\
% \hline
% \end{tabular}
% \caption{List of Alethe rules supported in our reconstruction.}
% \end{figure}

\begin{figure}[H]
\centering
\begin{tabular}{|ccc|}
\hline
\kw{ac\_simp} & \kw{and} & \kw{and\_neg} \\
\kw{and\_pos} & \kw{assume} & \kw{bind} \\
\kw{contraction} & \kw{distinct\_elim} & \kw{eq\_congruent} \\
\kw{eq\_congruent\_pred} & \kw{eq\_reflexive} & \kw{eq\_symmetric} \\
\kw{eq\_transitive} & \kw{equiv\_neg1} & \kw{equiv\_neg2} \\
\kw{equiv\_pos1} & \kw{equiv\_pos2} & \kw{equiv\_simplify} \\
\kw{false} & \kw{forall\_inst} & \kw{implies\_neg1} \\
\kw{implies\_neg2} & \kw{implies\_pos} & \kw{implies\_simplify} \\
\kw{ite\_neg1} & \kw{ite\_neg2} & \kw{ite\_pos1} \\
\kw{ite\_pos2} & \kw{ite\_simplify} & \kw{la\_generic} \\
\kw{lia\_generic} & \kw{not\_and} & \kw{not\_not} \\
\kw{not\_or} & \kw{not\_simplify} & \kw{not\_symm} \\
\kw{or} & \kw{or\_neg} & \kw{or\_pos} \\
\kw{refl} & \kw{reordering} & \kw{resolution} \\
\kw{sko\_exist} & \kw{sko\_forall} & \kw{subproof} \\
\kw{symm} & \kw{th\_resolution} & \kw{true} \\
\kw{xor\_neg1} & \kw{xor\_neg2} & \kw{xor\_pos1} \\
\kw{xor\_pos2} & & \\
\hline
\end{tabular}
\caption{List of Alethe rules supported in our reconstruction (alphabetical order, 3 columns).}
\end{figure}


$\kw{th\_resolution}$ and $\kw{resolution}$ denote the same rule.
The difference only distinguishes if the rule was introduced by the SAT solver or a theory solver.
The Alethe formats include 102 rules (excluding the rule \kw{hole}), and we currently support 52 of them. However, some rules in the format
are used exclusively by veriT, and do not appear in cvc5 proof traces. Among the unsupported rules, 14 are for linear arithmetic and 3 for bitvectors.

\begin{figure}[H]
\centering
\begin{tabular}{|cc|}
\hline
\kw{arith-elim-lt} & \kw{arith-geq-norm} \\
\kw{arith-geq-norm1} & \kw{arith-geq-norm2} \\
\kw{arith-geq-tighten} & \kw{arith-leq-norm} \\
\kw{bool-and-de-morgan} & \kw{bool-and-false} \\
\kw{bool-and-flatten} & \kw{bool-and-true} \\
\kw{bool-double-not-elim} & \kw{bool-eq-false} \\
\kw{bool-eq-true} & \kw{bool-impl-elim} \\
\kw{bool-impl-false1} & \kw{bool-impl-false2} \\
\kw{bool-impl-true2} & \kw{bool-impl-true2} \\
\kw{bool-or-de-morgan} & \kw{bool-or-false} \\
\kw{bool-or-flatten} & \kw{distinct-binary-elim} \\
\kw{eq-refl} & \kw{eq-symm} \\
\hline
\end{tabular}
\caption{List of RARE rules supported in our reconstruction (alphabetical order, 2 columns).}
\end{figure}